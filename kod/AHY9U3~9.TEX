(\emph{Dokument uppdaterat:} \today)

\hypertarget{struktur-av-system}{%
\section{Struktur av system}\label{struktur-av-system}}

\begin{figure}
\centering
\includegraphics{https://user-images.githubusercontent.com/30523857/98031352-a582f600-1e12-11eb-9e1f-11a99414f452.jpg}
\caption{Flödesdiagram av system}
\end{figure}

\begin{itemize}
\tightlist
\item
  \textbf{Hemsida} (interaktion med användarna -- dvs. uppladdning av
  mätdata som Excel-fil -- \emph{och} spelar upp musiken)
\item
  \textbf{Python-server} (tar emot användar-data, tolkar data och
  servrar)
\item
  \textbf{SuperCollider} (genererar musik)
\item
  \textbf{Webbradio} (dvs. \emph{DarkIce} och \emph{IceCast}, som
  strömmar ut musiken)
\end{itemize}

Kommunikation mellan Python-server och SuperCollider-patch sker
\emph{antingen} i realtid via OSC \textbf{eller} asynkront via
CSV-filer.

\hypertarget{blodsockervuxe4rden}{%
\section{Blodsockervärden}\label{blodsockervuxe4rden}}

Blodsocker mäts i mmol/L och varierar hos en icke-diabetiker mellan 4
och 6 mmol/L. Hos en diabetiker kan detta värde variera från under 1
till över 30 mmol/L, och Freestyle Libre-sensorn har ett spann på att
mäta från lägst 2,2 till 27,7 mmol/L (annars visar den ``\emph{LO}''
(sic) respektive ``\emph{HI}'' (sic)). Freestyle Libre-sensorn mäter
kontinuerligt var 15:e minut.

Att s.k. \emph{mappa} denna data till musikaliska parametrar är förstås
godtyckligt -- värdena i sig har ingen musikalisk mening -- och bör så
vara: det är helt enkelt min konstnärliga gärning som bestämmer hur de
förhåller sig till varandra. Även en bearbetad signal går att använda
för att styra musiken: interpolation (mellan de diskreta mätpunkterna),
variation (FFT, derivator, etc.), stokastiska egenskaper
(auto-korrelation etc), statistiska egenskaper (median, medel, etc.).
``\emph{Tid i målområdet}'' och liknande värden kan också vara
intressanta att använda, och har medicinsk betydelse.

Det som är viktigt i denna \emph{mappning} är dock att den gestaltade
datan -- dvs. musiken -- \textbf{inte} får avslöja något om den
underliggande eller bakomliggande (mät)datan. Dels är det en
integritetsfråga, som diskuteras vidare nedan, dels är det en
förutsättning för detta projekt: det existerar inga ``\emph{bra}'' eller
``\emph{dåliga}'' värden. Delningen av värdena är det viktiga, det är
via delningen som det gemensamma sker.

\hypertarget{integritet-delning-osv.}{%
\subsection{Integritet, delning osv.}\label{integritet-delning-osv.}}

\hypertarget{musik-supercollider-kod}{%
\section{Musik (SuperCollider-kod)}\label{musik-supercollider-kod}}

Varje instans av mätdata existerar som ett \emph{objekt} i musiken,
objekten har vissa attribut (såsom register, spatiell kodning, etc).
Koda gärna binauralt (kanske via \emph{Ambisonics}). Klassen har en
Osc-tolkarfunktion \textbf{eller} CSV-filläsare.

Använd \emph{Diabetessynth} som klangkälla? Kanske även andra Synthar.

Musiken ska vara deterministisk. Parametrarna styrs \emph{helt} av
blodsockervärdena.

\hypertarget{effektkedja}{%
\subsubsection{Effektkedja}\label{effektkedja}}

Använda effekter för bl.a. spatialitet (delay/reverb), förstärkning,
mixining och manipulation.

\hypertarget{klangkuxe4llor}{%
\subsubsection{Klangkällor}\label{klangkuxe4llor}}

Följande beskriver vilka ljudkällor (syntesmetoder) som kan tänkas
användas:

\begin{itemize}
\tightlist
\item
  \emph{Diabetessynth} (dvs. granulärsynth/wavetable-synth)
\item
  FM-synth/AM-synth
\item
  Annan granulär/sampler/wavetable-synth
\end{itemize}

\hypertarget{harmonicitet-spektralitet}{%
\subsubsection{Harmonicitet
(spektralitet)}\label{harmonicitet-spektralitet}}

Varje \emph{objekt} har följande attribut i förhållande till
spektralitet:

\begin{itemize}
\tightlist
\item
  Register
\item
  Tonart (bruksskala)
\item
  Stämning (\emph{renstämd/liksvävig})
\item
  Klangfärg (bestäms av mätdata?)
\end{itemize}

\hypertarget{temporalitet}{%
\subsubsection{Temporalitet}\label{temporalitet}}

Varje \emph{objekt} har följande attribut i förhållande till
temporalitet:

\begin{itemize}
\tightlist
\item
  Tempo
\end{itemize}

\hypertarget{spatialitet}{%
\subsubsection{Spatialitet}\label{spatialitet}}

Varje \emph{objekt} har följande attribut i förhållande till
spatialitet:

\begin{itemize}
\tightlist
\item
  Position
\item
  Bredd
\end{itemize}

\hypertarget{todo}{%
\section{TODO}\label{todo}}

\begin{enumerate}
\def\labelenumi{\arabic{enumi}.}
\tightlist
\item
  Kod för musik (skelettkod till en början)

  \begin{enumerate}
  \def\labelenumii{\arabic{enumii}.}
  \tightlist
  \item
    SuperCollider
  \item
    Python
  \end{enumerate}
\item
  Text till seminarium

  \begin{enumerate}
  \def\labelenumii{\arabic{enumii}.}
  \tightlist
  \item
    Skelett för layout av examenstext (\textbf{deadline 8/11})
  \item
    Litteraturstudie

    \begin{enumerate}
    \def\labelenumiii{\arabic{enumiii}.}
    \tightlist
    \item
      Låna: \emph{Det omätbaras renässans} av Jonna Bornemark
    \end{enumerate}
  \end{enumerate}
\end{enumerate}

\hypertarget{diverse}{%
\subsection{Diverse}\label{diverse}}

\begin{itemize}
\tightlist
\item[$\square$]
  Hantera \emph{lo}, \emph{hi}, och mg/dL (ist. för mmol/L).
\item[$\square$]
  Ska hemsida vara på svenska eller engelska?
\item[$\square$]
  Merge med ``idé''-textfil
\item[$\square$]
  Sätta upp GitHub (pages kanske t.o.m?)
\item[$\square$]
  Tänk på vilket register som ska motsvaras av vilken typ av
  ljudkälla\ldots{}
\item[$\square$]
  openFrameworks\ldots{} visualisering av mätdata?
\end{itemize}
