\documentclass[12pt]{article}
\usepackage[utf8]{inputenc}
\usepackage[swedish]{babel}
\usepackage{geometry}
\geometry{margin=2.5cm, top=3cm, headsep=1.75cm, footskip=1.25cm}

\setlength{\parindent}{0pt}%
\setlength{\parskip}{\baselineskip}%


\begin{document}
\section*{Essä}
\begin{center}
		\emph{\large{Karl Johannes Jondell, \today}}
\end{center}
När jag vid årsskiftet 2015/2016 fick ett ettårs-vikariat på Utvecklingsavdelningen på Sveriges Radio kunde jag inte ana att detta skulle betyda något särskilt för min musikaliska praktik. Även fast jag även på den tiden producerade och spelade in elektronisk musik i olika former, mynnade det aldrig ut till mer än korta projekt (oftast inte mer än åtta till sexton takter). Dessa var oftast beat-fokuserade, och tämligen \emph{konventionella}. 

När jag började min vikarietjänst på SR så var min uppgift att utforska binauralt ljud. Min erfarenhet av spatiellt ljud var begränsad till en delkurs av en kurs (som heter \emph{Audioteknik}) som jag läste på KTH. Visserligen hade jag tänkt \emph{spatiellt} i den mening att jag utforskat olika stereo-effekter, panorering, osv. Men aldrig förut hade jag beaktat detta som en \emph{central} musikalisk parameter. Jag hade haft kontakt med elektroakustisk musik tidigare på Fylkingen och Audiorama, men inte ägnat särskilt mycket tanke till hur rumsligheten påverkade musiken.

Till en början var jag helt ärligt skeptisk till behovet eller värdet av spatialisering (utöver den vanliga stereo-spatialiseringen): binauralt ljud kändes mycket som en plojgrej, en trend likt 3D-filmer på bio som kanske hade ett spännande nyhetsvärde men som sedan bara var onödigt \emph{krångligt} (behovet av komplicerade högtalarriggar/kvalitetshörlurar/3d-glasögon). Men i mitt arbete med binauralt ljud på SR så insåg jag hur mycket detaljer som går att lyfta fram i och med välgjord spatialisering: fler lager av ljud får helt plötsligt plats i ljudbilden! Kan tyckas vara självklart, men på något sätt klickade inte detta för mig förrän jag fick laborera med det själv. Min nyfikenhet för hur detta gick att använda i ett musikaliskt sammanhang föddes.

I samband med detta arbete fick jag utforska olika spatialiseringstekniker --- Ambisonics, binauralisering, WFS, objekt-orienterad panorering (''diskreta högtalare''), etc. --- och olika distributionsmetoder. Just Ambisonics är en teknik jag sedan fortsatt använda på KMH och som är särskilt användbart i Lilla Salen. Binauralisering är något jag kommer utforska i mitt examensprojekt (att sända en parallell binauraliserad version av musiken som jag genererar).

\newpage
\section*{Preliminär plan}
\begin{enumerate}
		\item ''\emph{Värden och värden: en studie av kollektiv(?) sonifiering}''
		\\Mitt examensarbete kommer låta diabetiker ladda upp sina blodsockervärden, vilket påverkar/styr en ström oändligt genererad musik.
\item Arbetet består i tre delar: dels bygga en hemsida där diabetiker kan ladda upp sina blodsockervärden, höra internetradiostation, läsa texter, interagera och ta del av projektet; dels bygga ett (SuperCollider-)program som genererar musiken; dels skriva om/problematisera/studera det större ämnet sonifiering (och även andra ämnen som berörs av projektet, bl.a.  upplevelserna av att ha diabetes). Dessa delar måste förstås även kopplas samman (SuperCollider-patchen beror av blodsockervärdena som delas till servern; patchen i sin tur strömmar musiken tillbaka till hemsidan). Det mer konstnärliga arbetet i att bestämma hur värdena ska tolkas i musiken och balansen av hur styrd musiken är av mina stilistiska preferenser och blodsockervärdena sker kontinuerligt. Jag önskar även föra en dialog med diabetiker i existerande diabetes-communities.
		\item Musiken ska presenteras på en internetradiostation som jag bygger. I en eventuell examenskonsert presenterar jag musiken som då spelas på radiostationen. Den är evig och formen bestäms av huruvida nån har delat sina blodsockervärden (som då triggar igång tolkningen av dessa, och som är ''aktiva'' i ca. ett dygn). Allt ljud produceras med SuperCollider (ev. andra ''hjälp-program'').
\end{enumerate}
\end{document}

