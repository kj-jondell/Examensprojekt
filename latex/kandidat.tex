\documentclass[11pt, twoside, a4paper]{article}
\usepackage{lipsum}
\usepackage{epigraph}
\usepackage{graphicx}
\usepackage[swedish]{babel}
\usepackage{natbib}
\usepackage[utf8]{inputenc}
\usepackage[margin=1in, bmargin=0.8in]{geometry}

\usepackage{csquotes}

\title{Värden och värden:\\
	\large en studie av kollektiv sonifiering
}
\author{Karl Johannes Jondell}
\date{\today}

\begin{document}

\maketitle

\begin{abstract}
Vad handlar detta projekt om?
\end{abstract}

% \begin{keyword}
% sonifiering \sep% 
% diabetes \sep%
% interaktion
% \end{keyword}

\tableofcontents

\section{Introduktion}
Detta projekt utgör ...

\subsection{Bakgrund}
Projekt från ettan, diabetes-synth, radiostation, etc.

\subsection{Sonifiering}
Sonifiering... \cite{bijsterveld_sonic_2019}. 
Smalley och spektromorfologin (olika ordningar av \emph{surrogacy},  gestaltandet av \emph{datan}).

Bearbetad data och orginaldata. Sensorfel.
\subsubsection{Det mätbara och det omätbara}
\cite{bornemark_det_2018}

\subsection{Diabetes}
Vad är diabetes? 

\subsubsection{Communities}

\subsubsection{Blodsockervärden}
Blodsocker mäts i mmol/L och varierar hos en icke-diabetiker mellan 4 och 6 mmol/L [källa]. Hos en diabetiker kan detta värde variera från under 1 till över 30 mmol/L, och Freestyle Libre-sensorn har ett spann på att mäta från lägst 2,2 till 27,7 mmol/L (annars visar den \emph{LO} respektive \emph{HI}). Freestyle Libre-sensorn mäter kontinuerligt var 15:e minut.

Att s.k. \emph{mappa} denna data till musikaliska parametrar är förstås godtyckligt --- värdena i sig har ingen musikalisk mening --- och bör så vara: det är helt enkelt min konstnärliga gärning som bestämmer hur de förhåller sig till varandra. Även en bearbetad signal går att använda för att styra musiken: interpolation (mellan de diskreta mätpunkterna), variation (FFT, derivator, etc.), stokastiska egenskaper (auto-korrelation etc), statistiska egenskaper (median, medel, etc.). ''\emph{Tid i målområdet}'' och liknande värden kan också vara intressanta att använda, och har medicinsk betydelse.

Det som är viktigt i denna \emph{mappning} är dock att den gestaltade datan --- dvs. musiken --- \textbf{inte} får avslöja något om den underliggande eller bakomliggande (mät)datan. Dels är det en integritetsfråga, som diskuteras vidare nedan, dels är det en förutsättning för detta projekt: det existerar inga \emph{bra} eller \emph{dåliga} värden. Själva delningen av värdena är det viktiga.


\section{Process}
Beskrivning/dokumentation av tekniken...

\begin{figure}[h!]
\centering
\includegraphics[width=\textwidth]{../media/flowchart.png}
\caption{Översiktsdiagram av system}
\end{figure}

\subsection{SuperCollider-system}
Varje instans av mätdata existerar som ett \emph{objekt} (motsvarande en ljudkälla, inte schaefferiansk) i musiken, objekten har vissa attribut (såsom register, spatiell kodning, etc). Även kodat binauralt (via \emph{Ambisonics}). Klassen har en Osc-tolkarfunktion (\textbf{eller} CSV-filläsare, om asynkron).

\section{Musiken}
Den konstnärliga friheten. Hur pass mycket kontroll som överlåtes till \enquote{serien} (i detta fall blodsockervärdet). Behöver musiken gestalta, spegla, estetisera erfarenheten som diabetiker? Eller vara intresseväckande, tillgänglig, \enquote{relaterbar}? 

\subsection{Rumslighet}
En \enquote{kör} av blodsockervärden, spatialiserade i nån mening för att ge en känsla av påverkan eller åverkan på musiken. 

Konsertupplevelse (i Lilla salen? spela ett utdrag ur liveströmmen...)

\subsection{Temporalitet}
Den tidsmässiga uppfattningen av musiken. En 24/7 livestream av musiken (hur utgörs lyssnadet? formen? \emph{Slow as possible}, \emph{Longplayer} och liknande...)

\subsection{Generativt}
Musiken är generativ. Serialism?

\section{Sammanfattning}
Lärdomar etc...

\bibliographystyle{agsm}
\bibliography{bibtex/kandidatarbete.bib}

\end{document}

